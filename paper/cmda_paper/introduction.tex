Data generation and collection in real-time is more pervasive than at any previous time due to technological progress in sensors, transmission, and data storage. This has resulted in enormous datasets, namely Big data, in various fields such as engineering, medical, physical, and social sciences. While effective modeling and simulation of such datasets can offer rich information for interpretation, prediction, monitoring, training, and decision making, their scope and complexity create significant design and implementation issues.\\
For instance, In the healthcare systems, continuous monitoring data have developed into a breakthrough for health diagnosis, subsequent treatment, and proactive health tracking \cite{athavale_biosignal_2017}. The recent advancements in wearable technology and monitoring systems have offered unseen opportunities for learning mechanisms of biomedical processes from temporal data as they provide data on physiological occurrences that indicate an individual’s health and comfort. 
However, it must be taken into account that the analysis of continuous monitoring data is highly affected by data quality issues and that even complex monitoring devices can monitor only a small subset of medical parameters involved in the process under consideration. Furthermore, the required analysis strategies for monitoring data vary significantly between monitoring a few parameters assessed on dense sampling rates (e.g., ECG, EEG, or glucose data) and monitoring several parameters over hours with low sampling frequency, as in ICU. 
On account of this, there is a significant demand for developing robust and easy-to-implement tools to facilitate continuous data wrangling and feature engineering.
In order to address this, we introduce the \textbf{CMDA} package to provide a comprehensive monitoring data analysis workflow.
\textbf{CMDA}, developed and built as a Python package, is a standard framework for time series data research, including temporal data importing, preprocessing, and feature extraction.